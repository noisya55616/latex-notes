\documentclass[12pt]{article}
\usepackage[utf8]{inputenc}
\usepackage{csquotes, amsmath, amssymb, graphicx, tikz, geometry, multicol}
\usepackage{wrapfig}
\geometry{margin=1in}

\setlength{\parindent}{0in}

\begin{document}

\begin{titlepage}

\begin{center}
    \Huge{Astronomy class 1: Elliptic orbits and celestial mechanics}
    
    \vspace{1in}
    
    \Large{Simon Wu}
    
    \Large{April 11, 2021}
    
    \vspace{1in}
    
    \Large{Olympiads School - Mr. Horbatsch}

    
\end{center}

\tableofcontents

\end{titlepage}

\section{Course overview}

\subsection{Textbook section}

The textbook that we are using in this course is \emph{Introduction to Astronomy} by \emph{Dr. Karina Kjaer}.
This class was based on material from \emph{Chapter 1} of the textbook.

\subsection{Fundamental theories of astronomy}

Astronomy is understanding things in our night sky.
To this end, we use two major theories: \emph{gravity} and \emph{quantum mechanics}.

\subsubsection{Gravity}

\begin{wrapfigure}{r}{0.2\textwidth}
    \centering
    \includegraphics[width=0.2\textwidth]{binary_pulsar}
    \label{fig:binarypulsar}
    \caption{Binary pulsar diagram.}
\end{wrapfigure}

Gravity is used to explain how objects in the night sky move.
In general, there is \emph{Newtonian gravity} (from Isaac Newton in the 1600's), and \emph{general relativity} (from Albert Einstein in the early 1900's).

Newtonian gravity is good for most things in the night sky.
However, not everything is that simple: some phenomena require the use of general relativity to accurately describe and explain.
General relativity is generally required for:
	
\begin{itemize}
	\item Black holes
	\item Cosmology (understanding the evolution of the entire universe)
	\item Binary pulsars (Figure \ref{fig:binarypulsar})
	\begin{itemize}
		\item Their orbits are shrinking ellipses and create spirals
		\item The effect requires general relativity to understand
		\begin{itemize}
			\item They output \emph{gravitational radiation} and lose energy in their orbits
		\end{itemize}
	\end{itemize}
	\item The orbit of Mercury (when high precision is needed)
	\begin{itemize}
		\item This is because Mercury is the planet closest to the Sun
		\item If you only need decent accuracy ($\sim 1\%$) then Newtonian gravity is sufficient
	\end{itemize}
\end{itemize}


\subsubsection{Quantum mechanics}

Quantum mechanics is mostly used to explain how stars shine (e.g. how the Sun shines).
It can predict the properties of emitted light from various atoms, and is also required to explain things like the stability of neutron stars and white dwarfs.

\newpage

\subsection{Length scales}

\begin{itemize}
	\item $10^{-15}$ meters: size of individual protons/neutrons.
	\item $10^{-10}$ meters: size of individual atoms.
	\item $1$ meter: normal stuff.
	\item $1$ AU, $10^{11}$ meters: the astronomical unit (AU), which is the Earth-Sun distance.
	\item $1$ pc, $1$ ly, $10^{16}$ meters: a light year (ly), parsec (pc, or parallax arc-second). About the distance to the nearest star, $\alpha$-Centauri
	\item $1$ kpc, $10^{19}$ meters: a kiloparsec. Magnitude of the size of galaxies e.g. Andromeda.
	\item $1$ Mpc, $10^{22}$ meters: a megaparsec. About the size of universe clusters.
	\item $1$ Gpc, $10^{25}$ meters: a gigaparsec. About the size of the observable universe. 
\end{itemize}

\section{Circular orbits}

\begin{wrapfigure}{r}{0.5\textwidth}
    \centering
    \includegraphics[width=0.5\textwidth]{circular_orbit}
    \label{fig:circularorbit}
    \caption{Circular orbit diagram.}
\end{wrapfigure}

Circular orbits can be parametrized as follows:

\[
x(t) = r\cos(\omega t)
\]

\[
y(t) = r\sin(\omega t)
\]

Of course, $v = r\omega$.
If the mass $M$ is considered to be much larger than $m$, the following equations can be derived for the orbital velocity:

\[
F_{net} = m\frac{v^2}{r} = \frac{GmM}{r^2}
\]

\[
v^2 = \frac{GM}{r}
\]

\newpage

\section{Elliptic orbits}

\subsection{Ellipse equation}

\begin{figure}[h]
    \centering
    \includegraphics[width=0.5\textwidth]{ellipse_equation}
    \label{fig:ellipseequation}
    \caption{Ellipse graph.}
\end{figure}

An ellipse can be defined with the following equation:

\[
\frac{x^2}{a^2} + \frac{y^2}{b^2} = 1
\]

Take $a,b>0$ and $a \geq b$.
$a$ is called the \emph{semi-major axis}, and $b$ is called the \emph{semi-minor axis}.
Notice that if $a = b$, it simplifies to a circle ($x^2 + y^2 = a^2$), of radius $a$.

\subsubsection{Eccentricity}

The eccentricity of an ellipse is represented with $\epsilon$.

\[
\epsilon = \sqrt{1 - b^2 / a^2} \in \mathbb{R}
\]

$b^2 / a^2 \leq 1$, so $0 \leq \epsilon \leq 1$.
Notice:

\begin{itemize}
	\item When $\epsilon = 0$, $b/a = 1$, and the ellipse reduces to a circle.
	\item When $\epsilon = 1$, $b/a = 0 \longrightarrow \lim_{b \to 0}$, and the ellipse collapses to a straight line segment.
\end{itemize}

\end{document}