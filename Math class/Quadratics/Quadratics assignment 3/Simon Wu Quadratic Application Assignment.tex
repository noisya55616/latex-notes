\documentclass[12pt]{article}
\usepackage[utf8]{inputenc}
\usepackage{csquotes, amsmath, amssymb, graphicx, tikz, geometry, multicol}
\geometry{margin=1in}

\setlength{\parindent}{0in}

\begin{document}

\begin{titlepage}

\begin{center}
    \Huge{Quadratic Applications Assignment}
    
    \vspace{0.5in}
    
    \Large{Simon Wu}
    
    \Large{\today}
    
    \vspace{0.5in}
    
    \Large{MPM2DE-B}

    
\end{center}

\section*{Problem statement}

\begin{displayquote}

Suppose a company charges $\$3.50$ per item and sells $52$ items daily. For every $\$0.50$ decrease in price per item, the daily sales increase by $3$ items.

\begin{enumerate}
\item Create an equation for a quadratic relation that models the daily revenue collected in terms of one independent variable. Remember to define your variables. Show your calculations and reasoning.
\item Use your equation from part $1$ to determine the maximum possible daily revenue. Show your calculations and reasoning.
\item Use your equation from part $1$ to determine the price per item that would yield daily revenue of $\$150$. Show your calculations and reasoning.
\end{enumerate}

\end{displayquote}

\tableofcontents

\end{titlepage}

\section{Modelling the situation with equations}

\subsection{Defining variables}

Let $r$ be the revenue generated in a day, and $m$ be the amount of money that the price of an item is decreased by, both in dollars.
Additionally, let $p$ be the price per item in dollars per item, and $n$ be the amount of items that the company sells in a day.

\subsection{Relating these values}

Firstly, we can observe that daily revenue is simply the product of the price per item and items sold per day.

\[
r = pn
\]

Next, we can observe that can be expressed in terms of the initial price of $\$3.50$ and the change in price $m$.

\[
p = 3.50 - m
\]

Finally, we can observe that the amount of items sold per day can also be expressed in terms of the initial daily sales volume of $52$ items and the decrease in price $m$ (sales increase by $3$ when the price decreases by $\$0.50$).

\[
n = 52 + \frac{3}{0.50}m = 52 + 6m
\]

\subsection{Final model of the situation}

Returning to the initial observation that $r = pn$ and the relations of $p$ and $n$ to the decrease in price of $m$, we can now express the daily revenue $r$ in terms of a single variable, $m$.

\[
r = (3.50 - m)(52 + 6m)
\]

I will take the liberty of expressing this equation in the proper factored form and the standard form.

\[
r = -6\left(m + \frac{26}{3}\right)(m - 3.50)
\]

\[
r = -6m^2 - 31m + 182
\]

\newpage

\section{Determining the maximum daily revenue}

The relation between the decrease in price $m$ and the daily revenue $r$ is quadratic.
It can also be observed that the quadratic relation will open downwards, because the coefficient in front of $m^2$ in the standard form of the relation is negative.

The maximum of the plot of revenue and decrease in price will therefore be at the vertex of the graph: the maximum daily revenue will be the optimum value, which will be the $r$-value of the vertex (considering points expressed as $(m, r)$ being analogous to $(x, y)$).

\subsection{Finding the optimum value}

Because I have an expression for the quadratic relation in the factored form, it is a very simple process to find the optimum value and therefore the maximum price: all I have to do is take the average of the roots of the quadratic relation and evaluate the relation with the price decrease yielded from that average.

The factored form of the quadratic relation is:

\[
r = -6\left(m + \frac{26}{3}\right)(m - 3.50)
\]

From this equation, it can be observed that the roots of the relation are $m = -\frac{26}{3}$ and $m = 3.50$.
The average of these values is $m_a = -\frac{31}{12}$: setting $m$ to this value will yield the maximum daily revenue possible.

I can therefore find the maximum daily revenue by simply evaluating the revenue relation with $m = -\frac{31}{12}$.

\[
r_{\text{max}} = -6\left(-\frac{31}{12}\right)^2 - 31\left(-\frac{31}{12}\right) + 182
\]

\[
r_{\text{max}} = \$\frac{5329}{29} \approx \boxed{\$222.04}
\]

\newpage

\section{Reaching exactly \$150 in daily revenue}

\subsection{Setting up an equation}

To set up this part of the problem, all we need to do is set $r = \$150$.
I will use the standard form of the relation here.

\[
r = 150 = -6m^2 - 31m + 182
\]

In order to actually find values of $m$ that will satisfy this, I will first rearrange the equation just a little.

\[
150 = -6m^2 - 31m + 182 \longrightarrow -6m^2 - 31m + 32 = 0 \longrightarrow \boxed{6m^2 + 31m - 32 = 0}
\]

Our new relation to work with is now $6m^2 + 31m - 32 = 0$.

\subsection{Determining appropriate values of $m$}

The best way to find values of $m$ that satisfy the relation is to factor the equation: we can evaluate when each of the terms is equal to $0$, because at least one of the terms of the factored equation will have to be equal to $0$ in order to satisfy the relation of the whole relation being equal to $0$.

Now, I actually don't think this equation can be factored normally with decomposition or just bashing factors at it.
However, I do know that there are solutions to this relation - $6m^2 + 31m - 32 = 0$ opens upwards and does pass through points below the $m$-axis (as evidenced by an $r$-intercept of $-32$ (because it is in standard form) that is less than $0$), so I know there must be values of $m$ that will satisfy this relation.

So, I will factor this equation using an extension the quadratic formula.
Given an equation of form $ax^2 + bx + c$, I know that the equation can be expressed in a factored form according to the following, because of the quadratic formula and the way it yields the roots:

\[
ax^2 + bx + c = a \left( x - \frac{-b + \sqrt{b^2 - 4ac}}{2a} \right) \left( x - \frac{-b - \sqrt{b^2 - 4ac}}{2a} \right)
\]

When you equate this to $0$:

\[
a \left( x - \frac{-b + \sqrt{b^2 - 4ac}}{2a} \right) \left( x - \frac{-b - \sqrt{b^2 - 4ac}}{2a} \right) = 0
\]

Because that entire thing on the left is equal to $0$, at least one of $a$, $\left( x - \frac{-b + \sqrt{b^2 - 4ac}}{2a} \right)$ and $\left( x - \frac{-b - \sqrt{b^2 - 4ac}}{2a} \right)$ must be equal to $0$, because they are all being multiplied together.

Looking at our equation, we can rewrite our equation in the same way, and come to the same conclusions about the values of the terms in that reqritten equation. $a$ of course cannot be $0$, because $a = 6$: therefore, at least one of $\left( m - \frac{-b + \sqrt{b^2 - 4ac}}{2a} \right)$ and $\left( m - \frac{-b - \sqrt{b^2 - 4ac}}{2a} \right)$ must be equal to $0$.

\newpage\

Therefore, we can simply rearrange a little to determine the values of $m$ that satisfy the equation.

\begin{align*}
m - \frac{-b + \sqrt{b^2 - 4ac}}{2a} &= 0\\
m &= \frac{-b + \sqrt{b^2 - 4ac}}{2a}
\end{align*}

Additionally:

\begin{align*}
m - \frac{-b - \sqrt{b^2 - 4ac}}{2a} &= 0\\
m &= \frac{-b - \sqrt{b^2 - 4ac}}{2a}
\end{align*}

From our expression $6m^2 + 31m - 32$, it can be seen that $a = 6$, $b = 31$ and $c=-32$. The decreases in price that would yield a daily revenue of exactly $\$150$ are yielded by those 2 equations above.
Calling the desired decreases in price $m_1$ and $m_2$:

\begin{align*}
m_1 &= \frac{-b + \sqrt{b^2 - 4ac}}{2a}\\
    &= \frac{-31 + \sqrt{31^2 - 4(6)(-32)}}{2(6)}\\
    &= \frac{-31 + \sqrt{1729}}{12}\\
    &\approx 0.882
\end{align*}

\begin{align*}
m_1 &= \frac{-b - \sqrt{b^2 - 4ac}}{2a}\\
    &= \frac{-31 - \sqrt{31^2 - 4(6)(-32)}}{2(6)}\\
    &= \frac{-31 - \sqrt{1729}}{12}\\
    &\approx -6.048
\end{align*}

Remembering that price $p = 3.5 - m$, the prices that will yield a daily revenue of exactly $\$150$ are $p_1 = 3.5 - m_1 = \boxed{\$2.62}$ and $p_2 = 3.5 - m_2 = \boxed{\$9.55}$

\end{document}