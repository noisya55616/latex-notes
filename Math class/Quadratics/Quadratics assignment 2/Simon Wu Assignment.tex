\documentclass[12pt]{article}
\usepackage[utf8]{inputenc}
\usepackage{csquotes, amsmath, amssymb}
\usepackage[margin=1in]{geometry}

\author{Simon Wu}
\date{March 2021}

\setlength{\parindent}{0in}

\begin{document}

\begin{titlepage}

\begin{center}
    \Huge{Quadratics assignment}
    
    \vspace{1in}
    
    \Large{Simon Wu}
    
    \Large{\today}
    
    \vspace{1in}
    
    \Large{MPM2DE-B}
    
\end{center}

\tableofcontents

\end{titlepage}

\section{My values}

My class list number is $26$. My values are therefore:

\begin{itemize}

\item $a=3$
\item $b=32$
\item $c=-11$
\item $u=36$
\item $v=484$
\item $w=18$

\end{itemize}

\section{Distributive property}

\subsection{Substituting values in}

\[
a(bx+c)^2 = 3(32x-11)^2
\]

\subsection{Expanding and simplifying}

Firstly, I prefer to begin by expanding exponents.
I'll start by using the distributive property to expand the square in this expression.

Knowing that by the distributive property, $(m+n)^2 = m^2 + 2mn + n^2$...

\[
3(32x-11)^2 = 3[(32x)^2 + 2(32x)(-11) + (-11)^2]
\]

Simplifying that...

\[
3(32x-11)^2 = 3(1024x^2 -704x + 121)
\]

Now, I can expand this further using the distributive property by multiplying every term inside of those brackets by $3$.

\[
3(32x-11)^2 = 3072x^2 -2112x + 363 \blacksquare
\]

This is as simple as it gets - no like terms to collect or anything.

\newpage

\section{Decomposition}

\subsection{Substituting values in}

\[
ax^2 + bx + c = 3x^2 + 32x - 11
\]

\subsection{The decomposition process}

\subsubsection{Finding some appropriate values}

Looking at the polynomial on our hands here, I know that there aren't any common factors to remove.
I will look for a pair of values such that they sum to $b=32$ and have a product of $a\times c = -33$.

Testing out some pairs of factors of $-33$, I can immediately notice that the numbers $33$ and $-1$ will have a product of $-33$ and a sum of $32$.

\subsubsection{Working with the trinomial}

Now, I can decompose the middle term of the trinomial into the values I found earlier to begin to factor it.

\[
3x^2 + 32x - 11 = 3x^2 - x + 33x - 11
\]

Notice that this very conveniently creates an opportunity to group some terms and begin to find some common factors.

\[
(3x^2 - x) + (33x - 11) = x(3x - 1) + 11(3x - 1)
\]

Rearranging this last expression a little, we find the fully factored trinomial.

\[
3x^2 + 32x - 11 = (3x - 1)(x + 11)\blacksquare
\]

\newpage

\section{Difference of squares}

\subsection{Substituting values in}

\[
ux^2 - v = 36x^2 - 484
\]

\subsection{Find the right method to factor this binomial}

Now, you didn't explicitly tell us what factoring method to use here, but I can immediately tell that there is a very easy way to factor this binomial.
This is because this binomial happens to be a difference of squares: both $36x^2$ and $484$ happen to be perfect squares! We can therefore very easily factor this binomial using the difference of squares method.

\subsection{Factoring the binomial with difference of squares}

Knowing that the factoring of a difference of squares takes the form of $m^2 - n^2 = (m+n)(m-n)$, we can observe that from the form of the equation, $m^2 = 36x^2$ and $n^2 = 484$.

We therefore know that $m = \pm \sqrt{36x^2} = \pm 6x$ and $n = \pm \sqrt{484} = \pm 22$.
I will take both $m$ and $n$ to be positive.

Therefore, our original binomial can be factored as follows, after pulling out some common factors: 

\[
36x^2 - 484 = (6x + 22)(6x - 22) = 4(3x + 11)(3x - 11)\blacksquare
\]

\newpage

\section{Completing the square}

\subsection{Substituting values in}

\[
y = ax^2 + wx + c \longrightarrow y = 3x^2 + 18x - 11
\]

\subsection{Completing the square for this trinomial}

Might as well get our hands dirty immediately.

\begin{align*} 
y &= 3x^2 + 18x - 11 \\
  &= 3(x^2 + 6x) - 11 \text{ (pulling $a$ out)}\\
  &= 3(x^2 + 6x + 9 - 9) - 11 \text{ (adding an appropriate value to get a perfect square)}\\
  &= 3(x^2 + 6x + 9) - 27 - 11 \text{ (pulling that value out of the bracket with dist. property)}\\
  &= 3(x + 3)^2 - 38\blacksquare \text{ (gathering the like terms on the right and simplifying the perfect square)}
\end{align*}

We have now completed the square.

\[
y = 3x^2 + 18x - 11 \longrightarrow \boxed{y = 3(x + 3)^2 - 38}
\]

\end{document}
