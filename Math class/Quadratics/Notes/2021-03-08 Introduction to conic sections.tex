\documentclass[12pt]{article}
\usepackage[utf8]{inputenc}
\usepackage{csquotes, amsmath, amssymb, graphicx, tikz, geometry, multicol, color, calc}
\geometry{margin=1in}

\setlength{\parindent}{0in}

\begin{document}

\begin{titlepage}

\begin{center}
    \Huge{Quadratics superdocument}
    
    \vspace{1in}
    
    \Large{Simon Wu}
    
    \Large{\today}
    
    \vspace{1in}
    
    \Large{MPM2DE-B}

    
\end{center}

\tableofcontents

\end{titlepage}

\section{Conic sections}

\section{Features of parabolas}

\section{Recognizing parabolas}

\subsection{Identifying features of a parabola}

\begin{center}

\begin{tikzpicture}[domain=-4:2]
    %\draw[very thin,color=gray] (-5,-5) grid (5,5);
    \draw[very thick,<->] (-5.2,0) -- (5.2,0) node[right] {$x$};
    \draw[very thick,<->] (0,-5.2) -- (0,5.2) node[above] {$y$};
    \draw[very thick,color=red] plot[id=eq1] function{(x + 3) * (x - 1)};

	\filldraw[black] (-1,-4) circle (2pt) node[anchor=north east] {Vertex};

	\filldraw[black] (-3,0) circle (2pt) node[anchor=south west] {Zero};
	\filldraw[black] (1,0) circle (2pt) node[anchor=south west] {Zero};

	\filldraw[black] (0,-3) circle (2pt) node[anchor=west] {Initial Value};

	\draw[thin, dashed] (-5,-4) -- (5,-4) node[anchor=west] {Optimal Value};
	\draw[thin, dashed] (-1,5) -- (-1,-5) node[anchor=north] {Axis of Symmetry};

\end{tikzpicture}

\end{center}

\newpage

\section{Standard equation form}

\subsection{The base parabola}

The most basic and simple equation for a parabola is as follows:

\[
\boxed{
y = x^2
}
\]

\subsection{The factored form}

The factored form would involve an equation that has a product.

\[
\boxed{
y = a(x - s)(x - t)
}
\]

Factored form is useful because it directly gives the roots, which will be $s$ and $t$.

\subsection{The standard form}

The standard form is as follows:

\[
\boxed{
y = ax^2 + bx + c
}
\text{ where } a,b,c \in \mathbb{R}
\]

You can express any degree of polynomial with this standard form and it's easy to do many different types of computation with it.

The parabola opens up if $a>0$ and down if $a<0$.
The parabola is vertically stretched is $|a| > 0$ and compressed if $0 < |a| < 1$.
$c$ gives the $y$-intercept of the parabola.

\subsection{The vertex form}

The vertex form is as follows:

\[
\boxed{
y = a(x - h)^2 + k
}
%\text{ where } a,b,c \in \mathbb{R}
\]

This equation form allows for a much easier visualization and drawing of the parabola from the equation, without any rearrangement.

The optimum value is $k$, the axis of symmetry lovation is $h$, and the vertex is located at $(h, k)$.
The value of $a$ determines the same properties of the parabola as it does in the standard form.
The parabola opens up if $a>0$ and down if $a<0$.
The parabola is vertically stretched is $|a| > 0$ and compressed if $0 < |a| < 1$.


\end{document}